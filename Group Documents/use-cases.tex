\documentclass[12pt]{article}

\pagestyle{empty}
\setcounter{secnumdepth}{2}

\topmargin=0cm
\oddsidemargin=0cm
\textheight=22.0cm
\textwidth=16cm
\parindent=0cm
\parskip=0.15cm
\topskip=0truecm
\raggedbottom
\abovedisplayskip=3mm
\belowdisplayskip=3mm
\abovedisplayshortskip=0mm
\belowdisplayshortskip=2mm
\normalbaselineskip=12pt
\normalbaselines
\usepackage{graphicx}
\usepackage{float}
\begin{document}

\vspace*{0.5in}
\centerline{\bf\Large Requirements Document}

\vspace*{0.5in}
\centerline{\bf\Large Team PB-PI}

\vspace*{0.5in}
\centerline{\bf\Large \today}

\vspace*{1.5in}
\begin{table}[htbp]
\caption{Team}
\begin{center}
\begin{tabular}{|r | c|}
\hline
Name & ID Number \\
\hline\hline
Alissa Bellerose & 27377320 \\
Sabrina D'Mello & 27739486 \\
Melanie Damilig & 40032420 \\
Tobi Decary-Larocque & 27407645 \\
Zain Farookhi & 26390684 \\
Giulia Gaudio & 27191766 \\
Jason Kalec & 40009464 \\
Damian Kazior & 40016168 \\
Johnny Mak & 40002140 \\
Philip Michael & 40004861 \\
Ramez Nicolas Nahas & 26718108 \\
Steven Tucci & 40006014 \\
Shunyu Wang & 40043915 \\
\hline
\end{tabular}
\end{center}
\end{table}

\clearpage

\section{System}

\subsection{Purpose}
The purpose of this document is to outline the requirements for the myMoney application. It provides direction and background to anyone involved in developing, testing and maintaining the system.

The intended audience of this document is decribed in table~\ref{tab:intended_audience}.

\begin{table}[htbp]
\caption{Intended audience}
\label{tab:intended_audience}
\begin{center}
  \begin{tabular}{|l|p{10cm}|}
      \hline
      Reader & Reason\\
      \hline\hline
      Users/Customers & Give feedback\\
      \hline
      System Developers & Understand the functionality and properties the application contains\\
      \hline
      Testers & Test the system\\
      \hline
      User Manual Writers & Source material for manuals\\
      \hline
      Project Team & Keep track of status of project\\
      \hline
  \end{tabular}
\end{center}
\end{table}

\subsection{Business Goals}
Our customers are money-conscious people who would benefit from a system that makes it easier to track how their money moves.

There currently is no efficient way for people to keep on top of their finances. To do it, they must rely on bank statements and receipts, and must learn to navigate through a variety of interfaces to keep track of different accounts. 

Our system makes it easier for people to control their finances, by allowing them for example to record cash deposits/withdrawals and track income/expenses across various accounts, all in one place.

\section{Domain Concepts}

\begin{figure}[h]
\centering
\caption{Domain model}
\includegraphics[width=160mm,natwidth=821,natheight=441]{model.jpg}
\label{fig:model}
\end{figure}
 
 
 
\begin{table}[H]
\caption{Main domain concepts}
\begin{center}
\begin{tabular}{|p{3cm}|p{14cm}|}
\hline
Concept & Description \\
\hline\hline
Student & Main user of the system. Student manually keeps the system up to date by entering every transaction performed during his day to day activities  \\
\hline
Bank & Student's bank. When student has accounts accross different banks, his accounts stay organized within the application \\
\hline
Account & Account under given bank. Account can be of type chequing, savings or investment\\
\hline
Transaction & Ever activiny that sudent performs (spending or depositing money) is recorded as a transaction \\
\hline
\end{tabular}
\end{center}
\end{table}

\section{System overview}


myMoney application is a simple, personal founds management system. It tracks user's deposits and expenses across different banks and and account within these banks. System is maintained by the user and is not connected to any bank account.

\section{Actors}

\begin{table}[H]
  \caption{User groups}
  \begin{center}
    \begin{tabular}{|l|p{8cm}|c|}
      \hline
      User Group & Description & Number of Users\\
      \hline\hline
      Young adults & Student or adult at an early stage of their career & 1\\
      \hline
    \end{tabular}
  \end{center}
\end{table}

\section{Functional Requirements}
\begin{table}[H]
\caption{Functionalities}
  \begin{center}
    \begin{tabular}{|l|p{10cm}|}
      \hline
      Field & Description\\
      \hline\hline
      \bf ID & \bf {F1}\\
      \hline
      Ver & 1\\
      \hline
      Feature & Transaction\\
      \hline
      Requirement & The system must be able to make a transaction from an account\\
      \hline
      Source & Team Brainstorm\\
      \hline
      Rationale & For a user to keep track of their budget, they must be able to record changes made within an account by adding transactions to said account\\
      \hline
      Priority & Must\\
      \hline
      Status & Proposed\\
      \hline
      Traces to use cases & Withdrawal \& Deposit Amount\\
      \hline
    \end{tabular}
  \end{center}
\end{table}
\begin{table}
  \begin{center}
    \begin{tabular}{|l|p{10cm}|}
      \hline
      \bf ID & \bf {F2}\\
      \hline
      Ver & 1\\
      \hline
      Feature & Transaction\\
      \hline
      Requirement & The user may only enter a numeric entry as input for the amount to deposit or withdraw\\
      \hline
      Source & User\\
      \hline
      Rationale & Monetary value is numerical, therefore we must ensure that input is valid, for system to accept the input\\
      \hline
      Priority & Must\\
      \hline
      Status & Proposed\\
      \hline
      Traces to use cases & Withdrawal \& Deposit Amount\\
      \hline
    \end{tabular}
  \end{center}
\end{table}
\begin{table}
  \begin{center}
    \begin{tabular}{|l|p{10cm}|}
      \hline
      \bf ID & \bf {F3}\\
      \hline
      Ver & 1\\
      \hline
      Feature & Transaction\\
      \hline
      Requirement & Transaction should require confirmation before being processed\\
      \hline
      Source & Team Brainstorm\\
      \hline
      Rationale & To ensure transaction is properly entered, system will prompt the user to confirm the transaction after inputting an amount\\
      \hline
      Priority & Must\\
      \hline
      Status & Proposed\\
      \hline
      Traces to use cases & Withdrawal \& Deposit Amount\\
      \hline
    \end{tabular}
  \end{center}
\end{table}

\section{Non-functional Requirements}

\begin{table}[H]
  \caption{Non-functional requirements}
  \begin{center}
    \begin{tabular}{|l|p{10cm}|}
      \hline
      Field & Description\\
      \hline\hline
      \bf ID & \bf {D1}\\
      \hline
      Ver & 1\\
      \hline
      Requirement &All documentation will be found within the source code in the repository\\
      \hline
      Source & Team Brainstorm\\
      \hline
      Rationale & All documents should be centrally located and accessible by all team members\\
      \hline
      Priority & Want\\
      \hline
      Status & Implemented\\
      \hline
      Traces to use cases & -\\
      \hline\hline
      \bf ID & \bf {S1}\\
      \hline
      Ver & 1\\
      \hline
      Requirement & User's financial information needs to be encrypted in database\\
      \hline
      Source & Organizer\\
      \hline
      Rationale & System will be handling sensitive information about the user's finances. Data such as account numbers will need to be encrypted for protection\\
      \hline
      Priority & Must\\
      \hline
      Status & Proposed\\
      \hline
      Traces to use cases & -\\
      \hline
    \end{tabular}
  \end{center}
\end{table}

\section{Use Cases}

\subsection{Overview}

\begin{figure}[h!]
  \centering
  \includegraphics[width=110mm,natwidth=565,natheight=466]{use_case_diagram.jpg}
  \caption{Use Case Diagram}
  \label{fig:use_case_diagram}
\end{figure}

\subsubsection{Use Case 1} \label{uc:1}

\noindent
{\bf ID}\\
UC-MWA-001\\
    
\noindent
{\bf Name}\\
Money Withdrawal.\\

\noindent
{\bf Goal}\\
The user withdraws an amount of money from the selected account.

\noindent
{\bf Actors}\\
Primary Actor - owner of the account.

\noindent
{\bf Precondition}\\
User is the owner of the account.

\noindent
{\bf Main Scenario}\\
\vspace*{-0.2in}
\begin{enumerate}
  \item Primary actor indicates to withdraw an amount from a selected account.
  \item System verifies the account exists.
  \item System prompts primary actor for the amount to withdraw.
  \item User enters amount to withdraw.
  \item System verifies that the account contains sufficient funds.
  \item System prompts user for confirmation.
  \item User confirms.
  \item System subtracts amount of money requested from selected account, confirms withdrawal.
  \item Use case ends succesfully.
\end{enumerate}

\noindent
    {\bf Exceptions}\\
    \vspace*{-0.2in}
\begin{itemize}
    \item[2a)] Account does not exist.
    \item[5a)] Account has an empty of negative balance.
    \item[5b)] Account contains insufficient funds.
\end{itemize}
    

\noindent
{\bf Postcondition}\\
The amount is subtracted from the selected account.

\noindent
{\bf Priority}\\
Critical.   

\noindent
{\bf Traces to Test Cases}\\
Add when test cases done.

\subsubsection{Use Case 2} \label{uc:2}

\noindent
{\bf ID}\\
UC - MDA - 002

\noindent
{\bf Name}\\
Deposit Money.

\noindent
{\bf Goal}\\
User succesfully deposits an amount of money the selected account.

\noindent
{\bf Actors}\\
Primary Actor - owner of the account.

\noindent
{\bf Precondition}\\
User is owner of the account.

\noindent
{\bf Main Scenario}\\
\vspace*{-0.2in}
\begin{enumerate}
  \item Primary actor indicates to deposit an amount to a selected account.
  \item System verifies that the account exists.
  \item System prompts user to enter amount to deposit.
  \item System verifies amount is valid.
  \item System adds amount of money deposited to selected account and confirms.
  \item Use case ends succesfully.
\end{enumerate}

\noindent
    {\bf Exceptions}\\
    \vspace{-0.2in}
    \begin{itemize}
    \item[2a)] Account does not exist.
    \item[4a)] Amount is zero or negative.
    \end{itemize}

\noindent
{\bf Postcondition}\\
The amount is added to the selected account.

\noindent
{\bf Priority}\\
Critical.

\noindent
{\bf Traces to Test Cases}\\
Add when test cases done.

\subsubsection{Use Case 3} \label{uc:3}

\noindent
{\bf ID}\\
UC - DBA - 003    

\noindent
{\bf Name}\\
Display Balance.

\noindent
{\bf Goal}\\
Display balance of chosen account to user.

\noindent
{\bf Actors}\\
Primary Actor - Owner of the account

\noindent
{\bf Precondition}\\
Account exists.

\noindent
{\bf Main Scenario}\\
\vspace*{-0.2in}
\begin{enumerate}
\item Primary Actor selects account to be displayed.
  \item System retrieves account information from database.
\item System displays balance.
\end{enumerate}

\noindent
{\bf Exceptions}\\
\vspace*{-0.2in}
\begin{itemize}
\item[1a)] Account does not exist.
\item[2a)] System cannot retrieve data for chosen account.
\end{itemize}

\noindent
{\bf Postcondition}\\
Account balance is displayed for user.

\noindent
{\bf Priority}\\
Critical.

\noindent
{\bf Traces to Test Cases}\\
Add when test cases done.

\section{Constraints}
\begin{table}[H]
\caption{System Constraints}
  \begin{center}
    \begin{tabular}{|l|p{10cm}|}
      \hline
      Field & Description\\
      \hline\hline
      ID & C1\\
      \hline
      Ver & 1\\
      \hline
      Constraint & The application must run on any well-known desktop operating system including Apple OS \& Windows\\
      \hline
      Source & Team Brainstorm\\
      \hline
      Rationale & Serve all users across different operating systems\\
      \hline
      Priority & Must\\
      \hline
      Status & Proposed\\
      \hline
      Traces to use cases & All use cases with a desktop interface\\
      \hline
    \end{tabular}
  \end{center}
\end{table}

\section{Solution ideas}
\begin{table}[H]
\caption{Solution Idea I}
  \begin{center}
    \begin{tabular}{|l|p{10cm}|}
      \hline
      Field & Description\\
      \hline\hline
      ID & SI1\\
      \hline
      Ver & 1\\
      \hline
      Solution idea & Total amount of withdrawals \& deposits could be displayed using charts\\
      \hline
      Source & Team Brainstorm\\
      \hline
      Rationale & When evaluating interesting budgeting applications, use of charts offers quick and clear overview of the user's financial situation\\
      \hline
      Traces to use cases & Display balance\\
      \hline
    \end{tabular}
  \end{center}
\end{table}

\section{Acronyms and Abbreviations}
\begin{table}[H]
  \begin{tabular}{l l}
    \bf{C} x & Constraint x\\
    \bf{ChAc} & Chequing Account\\
    \bf{DBA} & Display Balance\\
    \bf{F} x & Functionality x\\
    \bf{MDA} & Money Deposit\\
    \bf{MWA} & Money Withdrawal\\
    \bf{SavAc} & Savings Account\\
    \bf{SI} x & Solution Idea x\\
  \end{tabular}
\end{table}

\section{References}
We obtained an example User Requirements Document from the website http://www.soberit.hut.fi/T-76.115/05-06/ohjeet/template/requirements.html .

\appendix

\section{Description of File Format: Tasks}

Describe input file format.

\section{Description of File Format: Persons}

Describe output file format.

\end{document}
